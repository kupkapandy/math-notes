% !TeX root = ./main.tex

\chapter{Wielomiany}
\setcounter{section}{-1}
\section{Definicja jednomianu i wielomianu}

\begin{definition}
  Jednomian stopnia $n, \, n \in \N^+$, dla zmiennej rzeczywistej $x$, to wyrażenie o postaci
  $a \cdot x^n$, gdzie $a \neq 0$.
\end{definition}

\begin{example}
  Jednomian równy 0 to jednomian zerowy który nie posiada stopnia.
\end{example}

\begin{definition}
  Wielomian stopnia $n, \, n \in \N^+$, dla zmiennej rzeczywistej $x$, to wyrażenie o postaci:
  $$W(x)  = a_{n} \cdot x^{n} + a_{n - 1} \cdot x^{n - 1} + a_{n - 2} \cdot x^{n - 2} + \cdots + a_{2} \cdot x^{2} + a_{1} \cdot x + a_{0}$$
  gdzie, $a_i$ ($0 \leq i \leq n$) to ustalona liczba rzeczywista oraz $a_n \neq 0$.
\end{definition}

\begin{example}
  Suma współczynników wielomianu ($\sum_{i = 0}^{n}{a_i}$), wynosi $W(1)$.
\end{example}

\begin{example}
  Wielomian równy 0 to wielomian zerowy, $W(x) \equiv 0$.
\end{example}

\begin{example}
  Stopień wielomianu $W(x)$ to $\deg(W) = n$
\end{example}

\section{Wzory skróconego mnożenia}

\begin{theorem}
  Dla dowolnych $a$ i $b$ zachodzą:
  \begin{align*}
    \left( a + b \right)^3 &= a^3 + 3a^2b + 3ab^2 + b^3 \\
    \left( a - b \right)^3 &= a^3 - 3a^2b + 3ab^2 - b^3 \\
    a^3 + b^3 &= \left( a + b \right)\left( a^2 - ab + b^2 \right) \\
    a^3 - b^3 &= \left( a - b \right)\left( a^2 + ab + b^2 \right)
  \end{align*}
\end{theorem}

\begin{theorem}
  Dla dowolnych $a, b$ oraz $n > 1, \, n \in \N^+$, zachodzi:
  \[
    a^n - b^n = \left( a - b \right) \left( a^{n - 1} + a^{n - 2}b + a^{n - 3}b^2 + \cdots + a^2b^{n - 3} + ab^{n - 2} + b^{n - 1}\right) 
  \]
\end{theorem}

\begin{proof}
  Aby udowodnić twierdzenie wystarczy wykazać, że:
  \[
    \left( a - b \right) \left( a^{n - 1} + a^{n - 2}b + \cdots + ab^{n - 2} + b^{n - 1} \right) = a^n - b^n
  \]
  Rozwijając lewą strone równania otrzymujemy dwie sumy:
  \begin{align*}
    &a \sum_{k = 0}^{n - 1}{a^{n - 1 - k}b^{k}} = \sum_{k = 0}^{n - 1}{a^{n - k}b^{k}} = a^n + \sum_{k = 1}^{n - 1}{a^{n - k}b^k} \\
    &b \sum_{k = 0}^{n - 1}{a^{n - 1 - k}b^{k}} = \sum_{k = 0}^{n - 1}{a^{n - 1 - k}b^{k + 1}} = b^n + \sum_{k = 0}^{n - 2}{a^{n - 1 -k}b^{k + 1}}\\
  \end{align*}
  Łącząc powyższe sumy:
  \[
    a^n - b^n + \sum_{k = 1}^{n - 1}{a^{n - k}b^k} - \sum_{k = 0}^{n - 2}{a^{n - 1 - k}b^{k + 1}}
  \]
  Podstawiając $j = k + 1$ do pierwszej sumy zamiast $k$ otrzymujemy, że sumy są równe pozostawiając jedynie $a^n - b^n$.
\end{proof}
