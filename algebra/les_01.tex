% !TeX root = ./main.tex

\chapter{Wielomiany}
\setcounter{section}{-1}
\section{Definicja jednomianu i wielomianu}

\begin{definition}
  Jednomian stopnia $n, \, n \in \N^+$, dla zmiennej rzeczywistej $x$,
  to wyrażenie o postaci $a \cdot x^n$, gdzie $a \neq 0$.

  Jednomian stopnia zero to stała różna od zera. \\
  Jednomian zerowy to stała równa 0, nie ma określonego stopnia.
\end{definition}

\begin{example} $-\frac{3}{2} x^4$ to jednomian stopnia 4 \end{example}

\begin{definition}
  Wielomian stopnia $n, \, n \in \N^+$, dla liczby rzeczywistej $x$, to wyrażenie o postaci:
  \[
    a_{n} \cdot x^{n} + a_{n - 1} \cdot x^{n - 1} + a_{n - 2} \cdot x^{n - 2} + \dots + a_{2} \cdot x^{2} + a_{1} \cdot x + a_{0}
  \]
  Przy czym liczba $a_i$ to współczynnik wielomianu stopnia $n$ ($0 \leq i \leq n, \, a_n \neq 0$).
  Pojedyńcze jednomiany $a_n \cdot x^n, \, \dots , \, a_1 \cdot x, \, a_0$ to wyrazy wielomianu, gdzie $a_0$ to wyraz wolny.

  Wielomian stopnia zero to stała różna od zera, np. \(W(x) = -1\).
  Wielomian zerowy to liczba równa 0, \(W(x) \equiv 0\).

  Suma wszystkich współczynników wielomianu \(W(x)\) jest równa \(W(1)\). \\
  Stopień wielomianu \(W(x)\) oznaczany jest jako \(\deg\left(W\right) = n\)

  Jeżeli \(W\) i \(V\) to wielomiany to zachodzi:
  \[
    \deg\left(W \cdot V\right) = \deg(W) + \deg(V)
  \]
  \[
    \deg\left(W + V\right) = \max(\deg(W) + \deg(V))
  \]
\end{definition}

\begin{example} $W(x) = 4x^3 + -5x + 3, \, W(1) = 4 - 5 + 3 = 2$ \end{example}

\section{Wzory skróconego mnożenia}

\begin{theorem}
  Dla dowolnych $a, b$ zachodzą:
\begin{align*}
  \left(a + b\right)^3 = a^3 + 3a^2b + 3ab^2 + b^3 \\
  \left(a - b\right)^3 = a^3 - 3a^2b + 3ab^2 - b^3 \\
  a^3 + b^3 = \left(a + b\right)\left(a^2 - ab + b^2\right) \\
  a^3 - b^3 = \left(a - b\right)\left(a^2 + ab + b^2\right)
\end{align*}
\end{theorem}

\begin{theorem}
  Dla dowolnych $a, b$ oraz $n > 1, \, n \in \N$, zachodzi:
  \[
    a^n - b^n = \left(a - b\right)\left(a^{n - 1} + a^{n - 2}b + a^{n - 3}b^2 + \dots + a^2b^{n - 3} + ab^{n - 2} + b^{n - 1}\right)
  \]
  Ponadto, jeśli $a \neq b$, to dla dowolnego $n \in \N$, liczba $a - b$ jest dzielnikiem liczby $a^n - b^n$.
\end{theorem}
\begin{proof}
  Aby udowodnić twierdzenie wystarczy pokazać, że:
  \[(a - b)(a^{n-1} + a^{n-2}b + \cdots + b^{n-1}) = a^n - b^n\]
  Rozwijamy lewą strone równania:
    \[
      \left(a - b\right) \sum_{k = 0}^{n - 1}a^{n - 1 - k}b^k =
      a \sum_{k = 0}^{n - 1}a^{n - 1 - k}b^k - b \sum_{k = 0}^{n - 1}a^{n - 1 - k}b^k
    \]
  Pierwsza suma to:
  \[
    a\sum_{k = 0}^{n - 1}a^{n - 1 - k}b^{k} = \sum_{k = 0}^{n - 1}a^{n - k}b^{k} = \left(a^n + a^{n - 1}b + \dots + ab^{n - 1}\right)
  \]
  Natomiast druga suma to:
  \[
    b\sum_{k = 0}^{n - 1}a^{n - 1 - k}b^{k} = \sum_{k = 0}^{n - 1}a^{n - 1 - k}b^{k + 1} = \left(a^{n - 1}b + a^{n - 2}b^2 + \dots + b^n\right)
  \]
  Łączymy obie sumy i otrzymujemy:
  \[
    a^n + \sum_{k = 1}^{n - 1}a^{n - k}b^{k} - \sum_{k = 0}^{n - 2}a^{n - 1 - k}b^{k + 1} - b^n
  \]
  Podstawiając \(j = k + 1\) w drugiej sumie, otrzymujemy:
  \[
    \sum_{k = 0}^{n - 2}a^{n - 1 - k}b^{k + 1} = \sum_{j = 1}^{n - 1}a^{n - j}b^{j}
  \]
  Ponieważ \(\sum_{k = 1}^{n - 1}a^{n - k}b^k = \sum_{j = 1}^{n - 1}a^{n - j}b^k\), sumy się skracają, pozostawiając:
  \[
    a^n - b^n
  \]
\end{proof}


\begin{example}
  $a^{2k} - b^{2k} = a^{2k} - \left(-b\right)^{2k}$, gdzie $k \in N$,
\end{example}
