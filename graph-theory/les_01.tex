% !TeX root = ./main.tex

\section{Wprowadzenie i sformułowanie wyniku}
W 1960 roku Linnik pokazał, że istnieje nieskończenie wiele liczb pierwszych postaci $p = x^2 + y^2 + 1$,  
gdzie $x$ i $y$ są liczbami całkowitymi.  
Dokładniej, udowodnił on asymptotyczny wzór:
%
\begin{equation*}
  \sum_{p \leq X}{r(p - 1)}
    =
  \pi \prod_{p > 2}
  \left(
    1 + \frac{\chi_4(p)}{p(p - 1)}
  \right)
  \frac{X}{\log X}
    +
  \mathcal{O}
  \left(
    \frac{X (\log \log X)^7}{(\log X)^{1 + \theta_0}}
  \right)
\end{equation*}
%
gdzie $r(k)$ oznacza liczbę rozwiązań równania $k = x^2 + y^2$ w liczbach całkowitych,  
$\chi_4(k)$ to niegłówna (niepryncypalna) funkcja Dirichleta modulo 4, a
%
\begin{equation}
  \theta_0 = \frac{1}{2} - \frac{1}{4} e \log 2 = 0{,}0289 \dots
\end{equation}

\begin{theorem}
  Niech $f$ będzie funkcją różniczkowalną w każdym punkcie,  
  wtedy $f$ jest funkcją ciągłą.
\end{theorem}

\begin{definition}
  Niech $f$ będzie funkcją różniczkowalną w każdym punkcie,  
  wtedy $f$ jest funkcją ciągłą.
\end{definition}

\begin{lemma}
  Niech $f$ będzie funkcją różniczkowalną w każdym punkcie,  
  wtedy $f$ jest funkcją ciągłą.
\end{lemma}

\begin{proposition}
  Niech $f$ będzie funkcją różniczkowalną w każdym punkcie,  
  wtedy $f$ jest funkcją ciągłą.
\end{proposition}

\begin{corollary}
  Niech $f$ będzie funkcją różniczkowalną w każdym punkcie,  
  wtedy $f$ jest funkcją ciągłą.
\end{corollary}

\begin{remark}
  Niech $f$ będzie funkcją różniczkowalną w każdym punkcie,  
  wtedy $f$ jest funkcją ciągłą.
\end{remark}

\begin{example}
  Niech $f$ będzie funkcją różniczkowalną w każdym punkcie,  
  wtedy $f$ jest funkcją ciągłą.
\end{example}

\begin{proof}
  Niech $f$ będzie funkcją różniczkowalną w każdym punkcie,  
  wtedy $f$ jest funkcją ciągłą.
\end{proof}

\begin{proof}
  Niech $a$ i $b$ to wartości stałe
  oraz niech $d$ będzie najmniejszą wartością
  sumy $ax + by$, więc:
  \begin{equation}
    d = ak + bl.
  \end{equation}
  Aby udowodnić, że $d = \gcd(a, b)$ trzeba wykazać,
  że $d$ jest wspólnym dzielnikiem liczb $a$ oraz $b$
  i jest on jednocześnie największym takim dzielnikiem.

  \vspace{1em}

  \noindent\underline{Część 1.}

  Pokażemy, że $d$ jest wspólnym dzielnikiem liczb $a$ i $b$.
  Wiemy, że istnieją liczby $q$ oraz $r$ takie, że (Twierdzenie 2.):
  \[
    a = dq + r
  .\]
  gdzie, $0 \leq r < d$. Zmieniając zapis:
  \begin{align*}
    r &= a - dq \\ % add (1.1)
      &= a - (ak + bl)q \\
      &= a(1 - kq) + b(-lq) \\
  \end{align*}
  Jako, że $1 - kq$ oraz  $-lq$ to liczby całkowite, przedstawiliśmy
  $r$ jako $ax + by$. Ale jako, że $d$ to najmniejsza wartość wyrażenia
  $ax + by$ i $0 \leq r < d$, to $r = 0$.

  \vspace{1em}

  \noindent\underline{Część 2.}
  Pokażemy, że $d$ to największy wspólny dzielnik liczb $a$ i $b$.
  Niech $d'$ to inny wspólny dzielnik liczb $a$ i $b$, więc
  $d' \mid a$ oraz $d' \mid b$.
  \[
    d'n = a \quad \mathrm{i} \quad d'm = b
  \]
  gdzie $n, m$ to liczby całkowite.

  Podstawiając pod  $(1.1)$ otrzymujemy:
  \begin{align*}
    d &= ak + bl \\
      &= d'nk + d'ml \\
      &= d'(nk + ml) \\
  \end{align*}
  Zauważamy, że $d' = \frac{d}{nk + ml}$, otrzymując $d' \leq d$, co kończy dowód.
\end{proof}
