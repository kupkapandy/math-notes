% !TeX root = ./main.tex

\chapter{Podstawy}
\setcounter{section}{-1}
\section{Podzielność}

\iffalse
\begin{definition}
  ~ % To add empty line
  \begin{itemize}
    \item Liczba całkowita $n$ jest parzysta jeżeli $n = 2k$ dla jakiegoś $k$
    \item Liczba całkowita $n$ jest nieparzysta jeżeli $n = 2k + 1$ dla jakiegoś $k$
  \end{itemize}
\end{definition}
\fi

\begin{definition}
  Liczba całkowita $b$ jest podzielna przez liczbe całkowitą $a$,
  wtedy gdy $b = ak$, dla jakieś liczby całkowitej $k$.
  Gdy $a$ dzieli $b$ to $a \mid b$,
  natomiast gdy $a$ nie dzieli $b$ to $a \nmid b$.
\end{definition}

\begin{theorem}
  Niech $a, b, c$ to liczby całkowite. Jeżeli $a \mid b$ i $b \mid c$, to $a \mid c$.
\end{theorem}

\begin{proof}
  Niech $b = ak$ i $c = bl$.
  Zatem $c = akl$, gdzie $kl$ to liczba całkowita.
  Więc liczba $a$ dzieli liczbę $c$ ($a \mid c$).
\end{proof}

\begin{theorem}
  Dla wszystkich liczb całkowitych $a$ i $m$ gdzie $m > 0$,
  istnieją dwie liczby całkowite $q$ i $r$, takie że:
  \[
    a = mq + r
  \]
  gdzie $0 \leq r < m$.
\end{theorem}

\section{Największy wspólny dzielnik (GCD)}

\begin{definition}
  Niech $a$ i $b$ to liczby całkowite.
  Jeżeli $c \mid a$ i $c \mid b$,
  to $c$ jest wspólnym dzielnikiem liczb $a$ i $b$.

  Największy wspólny dzielnik
  liczb $a$ i $b$ to największa liczba $d$,
  taka która $d \mid a$ i $d \mid b$.
  Oznacza się ją $\nwd(a, b)$, $\gcd(a, b)$.
\end{definition}

\begin{theorem}
  (\textit{Tożsamość Bézouta}).
  Jeżeli a i b to dodatnie liczby całkowite,
  to istnieją liczby całkowite $k$ i $l$, takie że:
  \[
    \gcd(a, b) = ak + bl
  \]
\end{theorem}

\begin{proof}
  Niech $a$ i $b$ to wartości stałe
  oraz niech $d$ będzie najmniejszą wartością
  sumy $ax + ay$, więc:
  \begin{equation}
    d = ak + bl.
  \end{equation}
  Aby udowodnić, że $d = \gcd(a, b)$ trzeba wykazać,
  że $d$ jest wspólnym dzielnikiem liczb $a$ oraz $b$
  i jest on jednocześnie największym takim dzielnikiem.

  \vspace{1em}

  \noindent\underline{Część 1.}
  Pokażemy, że $d$ jest wspólnym dzielnikiem liczb $a$ i $b$.
  Wiemy, że istnieją liczby $q$ oraz $r$ takie, że (Twierdzenie 2.):
  \[
    a = dq + r
  .\]
  gdzie, $0 \leq r < d$. Zmieniając zapis:
  \begin{align*}
    r &= a - dq \\ % add (1.1)
      &= a - (ak + bl)q \\
      &= a(1 - kq) + b(-lq) \\
  \end{align*}

  \vspace{1em}

  \noindent\underline{Część 2.}
  Wykazanie, że $d$ to największy wspólny dzielnik $a$ oraz $b$.

\end{proof}
