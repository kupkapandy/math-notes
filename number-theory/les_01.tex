% !TeX root = ./main.tex

\chapter{Podstawy}
\setcounter{section}{-1}
\section{Podzielność}

\iffalse
\begin{definition}
  ~ % To add empty line
  \begin{itemize}
    \item Liczba całkowita $n$ jest parzysta jeżeli $n = 2k$ dla jakiegoś $k$
    \item Liczba całkowita $n$ jest nieparzysta jeżeli $n = 2k + 1$ dla jakiegoś $k$
  \end{itemize}
\end{definition}
\fi

\begin{definition}
  Liczba całkowita $b$ jest podzielna przez liczbe całkowitą $a$,
  wtedy gdy $b = ak$, dla jakieś liczby całkowitej $k$.
  Gdy $a$ dzieli $b$ to $a \mid b$,
  natomiast gdy $a$ nie dzieli $b$ to $a \nmid b$.
\end{definition}

\begin{theorem}
  Niech $a, b, c$ to liczby całkowite. Jeżeli $a \mid b$ i $b \mid c$, to $a \mid c$.
\end{theorem}

\begin{proof}
  Niech $b = ak$ i $c = bl$.
  Zatem $c = akl$, gdzie $kl$ to liczba całkowita.
  Więc liczba $a$ dzieli liczbę $c$ ($a \mid c$).
\end{proof}

\begin{theorem}
  Dla wszystkich liczb całkowitych $a$ i $m$ gdzie $m > 0$,
  istnieją dwie liczby całkowite $q$ i $r$, takie że:
  \[
    a = mq + r
  \]
  gdzie $0 \leq r < m$.
\end{theorem}

\section{Największy wspólny dzielnik (GCD)}

\begin{definition}
  Niech $a$ i $b$ to liczby całkowite.
  Jeżeli $c \mid a$ i $c \mid b$,
  to $c$ jest wspólnym dzielnikiem liczb $a$ i $b$.

  Największy wspólny dzielnik
  liczb $a$ i $b$ to największa liczba $d$,
  taka która $d \mid a$ i $d \mid b$.
  Oznacza się ją $\nwd(a, b)$, $\gcd(a, b)$.
\end{definition}

\begin{theorem}
  (\textit{Tożsamość Bézouta}).
  Jeżeli a i b to dodatnie liczby całkowite,
  to istnieją liczby całkowite $k$ i $l$, takie że:
  \[
    \gcd(a, b) = ak + bl
  \]
\end{theorem}

\begin{proof}
  Niech $a$ i $b$ to wartości stałe
  oraz niech $d$ będzie najmniejszą wartością
  sumy $ax + by$, więc:
  \begin{equation}
    d = ak + bl.
  \end{equation}
  Aby udowodnić, że $d = \gcd(a, b)$ trzeba wykazać,
  że $d$ jest wspólnym dzielnikiem liczb $a$ oraz $b$
  i jest on jednocześnie największym takim dzielnikiem.

  \vspace{1em}

  \noindent\underline{Część 1.}
  Pokażemy, że $d$ jest wspólnym dzielnikiem liczb $a$ i $b$.
  Wiemy, że istnieją liczby $q$ oraz $r$ takie, że (Twierdzenie 2.):
  \[
    a = dq + r
  .\]
  gdzie, $0 \leq r < d$. Zmieniając zapis:
  \begin{align*}
    r &= a - dq \\ % add (1.1)
      &= a - (ak + bl)q \\
      &= a(1 - kq) + b(-lq) \\
  \end{align*}
  Jako, że $1 - kq$ oraz  $-lq$ to liczby całkowite, przedstawiliśmy
  $r$ jako $ax + by$. Ale jako, że $d$ to najmniejsza wartość wyrażenia
  $ax + by$ i $0 \leq r < d$, to $r = 0$.

  \vspace{1em}

  \noindent\underline{Część 2.}
  Pokażemy, że $d$ to największy wspólny dzielnik liczb $a$ i $b$.
  Niech $d'$ to inny wspólny dzielnik liczb $a$ i $b$, więc
  $d' \mid a$ oraz $d' \mid b$.
  \[
    d'n = a \quad \mathrm{i} \quad d'm = b
  \]
  gdzie $n, m$ to liczby całkowite.

  Podstawiając pod  $(1.1)$ otrzymujemy:
  \begin{align*}
    d &= ak + bl \\
      &= d'nk + d'ml \\
      &= d'(nk + ml) \\
  \end{align*}
  Zauważamy, że $d' = \frac{d}{nk + ml}$, otrzymując $d' \leq d$, co kończy dowód.
\end{proof}

\begin{definition}
  Niech $a, r$ i $m$ to liczby całkowite. Mówimy, że liczba $a$ jest przystająca
  do liczby $r$ modulo $m$, gdy $m \mid a - r$.

  Innymi słowy: liczby $a$ i $r$
  mają tę samą resztę z dzielenia przez $m$, czyli:
  \[
    a \equiv r \pmod{m}
  .\]
\end{definition}

\begin{theorem}
  (\textit{Prawo anulowania}).
  Niech $a, \, b, \, k$ i $m$ to liczby całkowite, gdzie $k \neq 0$.
  Jeżeli $ak \equiv bk \pmod{m}$ i $\gcd(k, m) = 1$, to $a \equiv b \pmod{m}$.
\end{theorem}

\begin{theorem}
  (\textit{Małe twierdzenie Fermata}). Jeżeli $a$ to liczba całkowita i
  $p$ to liczba pierwsza która nie dzieli $a$, to \[
    a^{p - 1} \equiv 1 \pmod{p}
  .\]
\end{theorem}
